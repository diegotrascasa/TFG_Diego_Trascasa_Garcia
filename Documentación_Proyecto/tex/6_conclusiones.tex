\capitulo{6}{Conclusiones}

El desarrollo del sistema de monitoreo cardíaco en tiempo real ha logrado cumplir con los objetivos establecidos al inicio del proyecto. Este proyecto ha permitido la creación de una herramienta innovadora, efectiva y económica para la supervisión continua de la actividad cardíaca.

Desde la implementación inicial hasta los resultados obtenidos, el sistema ha demostrado ser una solución viable y práctica para el monitoreo cardíaco, proporcionando datos en tiempo real que pueden ser utilizados tanto por pacientes como por profesionales de la salud. La elección del módulo AD8232, debido a su alta precisión y capacidad para manejar señales ECG en condiciones ruidosas, ha sido un factor clave en la eficacia del sistema.

\section{Aspectos relevantes}

Uno de los mayores desafíos enfrentados durante el desarrollo del proyecto fue establecer una conexión inalámbrica eficiente entre el dispositivo Arduino y la aplicación web. Inicialmente, se consideró la implementación de un módulo Bluetooth o WiFi, sin embargo, se optó por una conexión USB debido a problemas de transmisión de los datos en las pruebas realizadas con la conexión inalámbrica. Este cambio forzado, aunque limitó la movilidad, garantizó una transmisión de datos constante y fiable.

El desarrollo de la interfaz web con Streamlit presentó una serie de desafíos debido a la falta de experiencia realizando este tipo de trabajos pero observando algunos ejemplos y documentandome sobre su uso se consiguió un resultado final bueno y sencillo. La implementación de una base de datos SQLite y el uso de modelos de machine learning para la predicción de anomalías cardíacas representaron importantes avances que enriquecieron el proyecto.

\section{Limitaciones y desafíos}

El proyecto encontró varias limitaciones, especialmente en términos de la calidad de los datos de entrenamiento para el modelo de predicción. La precisión del modelo de machine learning está directamente influenciada por la calidad y diversidad de los datos utilizados, lo que resalta la necesidad de mejorar los conjuntos de datos de entrenamiento en futuras implementaciones.

Además, aunque la conexión USB proporcionó una solución estable, la falta de movilidad sigue siendo una limitación significativa. Retomar la investigación para implementar una conexión inalámbrica efectiva sigue siendo una prioridad para mejorar la comodidad y la usabilidad del sistema.


