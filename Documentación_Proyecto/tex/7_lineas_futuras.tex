\capitulo{7}{Líneas de trabajo futuras}

Aunque el proyecto ha logrado sus objetivos, existen oportunidades para futuras mejoras e implementaciones en diversas áreas.

En primer lugar, es crucial mejorar la precisión del modelo de predicción. Para ello, es necesario aumentar la calidad y cantidad de los datos de entrenamiento, utilizando bases de datos más extensas y diversificadas, lo que permitirá mejorar la precisión del modelo de machine learning. Además, se debe permitir que el modelo pueda clasificar y añadir etiquetas de patologías directamente, no solo clasificar los tipos de latido de cada segmento.

Otra área de mejora es la implementación de una conexión inalámbrica estable. Es necesario retomar la investigación y desarrollo para implementar un módulo Bluetooth o WiFi que permita la comunicación inalámbrica entre el dispositivo de monitoreo y la aplicación web. Esto mejorará la comodidad y movilidad del usuario, eliminando la dependencia de la conexión USB.

La ampliación de las funcionalidades de la aplicación web también es una prioridad. Integrar herramientas adicionales para el análisis de los datos de ECG, como la detección automática de diferentes tipos de arritmias, y generar informes detallados para los profesionales de la salud facilitará el diagnóstico y seguimiento del paciente.

En cuanto al hardware, se deben desarrollar versiones más compactas y ergonómicas del dispositivo de monitoreo. Además, incluir opciones para la alimentación mediante baterías recargables aumentará la autonomía del sistema.

Finalmente, es esencial implementar medidas adicionales para asegurar la privacidad y seguridad de los datos de los pacientes. Cumplir con normativas y estándares internacionales de protección de datos garantizará la confidencialidad de la información médica.


