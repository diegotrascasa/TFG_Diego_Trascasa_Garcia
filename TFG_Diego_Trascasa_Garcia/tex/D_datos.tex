\apendice{Descripción de adquisición y tratamiento de datos}

\section{Descripción formal de los datos}

En esta sección se detalla la estructura y formato de los datos empleados que se han usado para el proyecto y los que se usan durante su funcionamiento, que incluye tanto la recopilación de datos a través del sensor AD8232 como el uso de datos de entrenamiento y pruebas proporcionados por Kaggle.

\subsection{Datos Recogidos por el sensor AD8232}
Los datos capturados por el sensor AD8232 son fundamentales para el monitoreo en tiempo real de la actividad cardíaca. Son datos de tipo flotante (float) que representan las señales eléctricas del corazón capturadas en mV (milivoltios). Estos datos se recogen en tiempo real y se almacenan para su posterior análisis, sabiendo que se graban cada 10ms.

\subsection{Datos de entrenamiento y test de Kaggle}
Para el desarrollo y validación del modelo de predicción, se utilizan datos de entrenamiento y test obtenidos de Kaggle (se pueden encontrar en el \href{https://github.com/diegotrascasa/TFG_Diego_Trascasa_Garcia}{repositorio de GitHub}, en la carpeta DataECG). Son conjuntos de datos que contienen segmentos de señales de ECG etiquetados con el tipo de latido. Cada dato está en formato de punto flotante y clasificado según un esquema de etiquetas predefinido, que permite la identificación de diversos tipos de latido.

\subsection{Datos generados por el modelo de predicción}
Una vez procesados los datos a través del modelo de predicción, se generan nuevas etiquetas que clasifican cada ciclo cardíaco detectado. Cada etiqueta generada por el modelo es de tipo categórico (string), indicando el tipo de latido cardíaco. Estas predicciones se almacenan junto con su respectiva fecha y hora en la base de datos.

\subsection{Almacenamiento de datos en la base de datos}
Los datos recogidos y generados se almacenan en una base de datos estructurada, que incluye:

\begin{itemize}
    \item \textbf{ID del registro}: Clave primaria de tipo entero (int) que identifica de manera única cada conjunto de datos.
    \item \textbf{Timestamp}: Sello temporal de la captura de los datos.
    \item \textbf{Datos de ECG}: Almacenados en formato de punto flotante.
    \item \textbf{Etiquetas de predicción}: Almacenadas en formato categórico.
\end{itemize}




\section{Descripción clínica de los datos}

La interpretación clínica de los datos recolectados es fundamental para el diagnóstico y seguimiento de condiciones cardíacas.

Las señales de ECG proporcionan una representación gráfica de la actividad eléctrica del corazón. Cada ciclo cardíaco registrado en la señal incluye diferentes ondas (P, QRS, T) que representan distintas fases del ciclo cardíaco. Anomalías en estas ondas pueden indicar la presencia de arritmias, isquemia, o otras patologías cardíacas.
La frecuencia cardíaca, medida en latidos por minuto (BPM), es un indicador vital de la salud cardíaca. Valores anormalmente altos (taquicardia) o bajos (bradicardia) pueden ser signos de disfunciones cardíacas que requieren atención médica.
La clasificación automática de los latidos en tipos específicos (e.g., normales, ectopia supraventricular, ectopia ventricular) ayuda a los médicos a identificar patrones de comportamiento anómalos del corazón, facilitando el diagnóstico temprano y el tratamiento adecuado de condiciones cardíacas.

