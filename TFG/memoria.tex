\documentclass[a4paper,12pt,twoside]{memoir}

% Castellano
\usepackage[spanish,es-tabla]{babel}
\selectlanguage{spanish}
\usepackage[utf8]{inputenc}
\usepackage[T1]{fontenc}
\usepackage{lmodern} % Scalable font
\usepackage{microtype}
\usepackage{placeins}

\RequirePackage{booktabs}
\RequirePackage[table]{xcolor}
\RequirePackage{xtab}
\RequirePackage{multirow}

% Links
\PassOptionsToPackage{hyphens}{url}\usepackage[colorlinks]{hyperref}
\hypersetup{
	allcolors = {red}
}

% Ecuaciones
\usepackage{amsmath}

% Rutas de fichero / paquete
\newcommand{\ruta}[1]{{\sffamily #1}}

% Párrafos
\nonzeroparskip

% Huérfanas y viudas
\widowpenalty100000
\clubpenalty100000

\let\tmp\oddsidemargin
\let\oddsidemargin\evensidemargin
\let\evensidemargin\tmp
\reversemarginpar

% Imágenes

% Comando para insertar una imagen en un lugar concreto.
% Los parámetros son:
% 1 --> Ruta absoluta/relativa de la figura
% 2 --> Texto a pie de figura
% 3 --> Tamaño en tanto por uno relativo al ancho de página
\usepackage{graphicx}

\newcommand{\imagen}[3]{
	\begin{figure}[!h]
		\centering
		\includegraphics[width=#3\textwidth]{#1}
		\caption{#2}\label{fig:#1}
	\end{figure}
	\FloatBarrier
}







\graphicspath{ {./img/} }

% Capítulos
\chapterstyle{bianchi}
\newcommand{\capitulo}[2]{
	\setcounter{chapter}{#1}
	\setcounter{section}{0}
	\setcounter{figure}{0}
	\setcounter{table}{0}
	\chapter*{#2}
	\addcontentsline{toc}{chapter}{#2}
	\markboth{#2}{#2}
}

% Apéndices
\renewcommand{\appendixname}{Apéndice}
\renewcommand*\cftappendixname{\appendixname}

\newcommand{\apendice}[1]{
	%\renewcommand{\thechapter}{A}
	\chapter{#1}
}

\renewcommand*\cftappendixname{\appendixname\ }

% Formato de portada

\makeatletter
\usepackage{xcolor}
\newcommand{\tutor}[1]{\def\@tutor{#1}}
\newcommand{\tutorb}[1]{\def\@tutorb{#1}}

\newcommand{\course}[1]{\def\@course{#1}}
\definecolor{cpardoBox}{HTML}{E6E6FF}
\def\maketitle{
  \null
  \thispagestyle{empty}
  % Cabecera ----------------
\begin{center}
  \noindent\includegraphics[width=\textwidth]{cabeceraSalud}\vspace{1.5cm}%
\end{center}
  
  % Título proyecto y escudo salud ----------------
  \begin{center}
    \begin{minipage}[c][1.5cm][c]{.20\textwidth}
        \includegraphics[width=\textwidth]{escudoSalud.pdf}
    \end{minipage}
  \end{center}
  
  \begin{center}
    \colorbox{cpardoBox}{%
        \begin{minipage}{.8\textwidth}
          \vspace{.5cm}\Large
          \begin{center}
          \textbf{TFG del Grado en Ingeniería de la Salud}\vspace{.6cm}\\
          \textbf{\LARGE\@title{}}
          \end{center}
          \vspace{.2cm}
        \end{minipage}
    }%
  \end{center}
  
    % Datos de alumno, curso y tutores ------------------
  \begin{center}%
  {%
    \noindent\LARGE
    Presentado por \@author{}\\ 
    en Universidad de Burgos\\
    \vspace{0.5cm}
    \noindent\Large
    \@date{}\\
    \vspace{0.5cm}
    %Tutor: \@tutor{}\\ % comenta el que no corresponda
    Tutor: \@tutor{} \\
  }%
  \end{center}%
  \null
  \cleardoublepage
  }
\makeatother

\newcommand{\nombre}{Diego Trascasa García}
\newcommand{\nombreTutor}{Guirguis Zaki Guirguis Abdelmessih} 
\newcommand{\dni}{71567758S} 

% Datos de portada
\title{Aplicación web de monitoreo cardíaco en tiempo real con análisis predictivo de tipos de latidos en ECG.}
\author{\nombre}
\tutor{\nombreTutor}
\tutorb{\nombreTutorb}
\date{\today}


\begin{document}

\maketitle


\newpage\null\thispagestyle{empty}\newpage

%%%%%%%%%%%%%%%%%%%%%%%%%%%%%%%%%%%%%%%%%%%%%%%%%%%%%%%%%%%%%%%%%%%%%%%%%%%%%%%%%%%%%%%%
\thispagestyle{empty}


\noindent\includegraphics[width=\textwidth]{cabeceraSalud}\vspace{1cm}

\noindent D. \nombreTutor, profesor del departamento de departamento, área de área.

\noindent Expone:

\noindent Que el alumno D. \nombre, con DNI \dni, ha realizado el Trabajo final de Grado en Ingeniería de la Salud titulado título Aplicación Web de Monitoreo Cardíaco en Tiempo Real con Análisis Predictivo de Tipos de Latidos en ECG. 

\noindent Y que dicho trabajo ha sido realizado por el alumno bajo la dirección del que suscribe, en virtud de lo cual se autoriza su presentación y defensa.

\begin{center} %\large
En Burgos, {\large \today}
\end{center}

\vfill\vfill\vfill

% Author and supervisor
\begin{minipage}{0.45\textwidth}
\begin{flushleft} %\large
Vº. Bº. del Tutor:\\[2cm]
D. \nombreTutor
\end{flushleft}
\end{minipage}
\hfill
\begin{minipage}{0.45\textwidth}
\begin{flushleft} %\large
\end{flushleft}
\end{minipage}
\hfill

\vfill

% para casos con solo un tutor comentar lo anterior
% y descomentar lo siguiente
%Vº. Bº. del Tutor:\\[2cm]
%D. nombre tutor


\newpage\null\thispagestyle{empty}\newpage




\frontmatter

% Abstract en castellano
\renewcommand*\abstractname{Resumen}
\begin{abstract}
El monitoreo de la actividad cardíaca es crucial para la detección temprana y el manejo adecuado de diversas afecciones cardíacas. Sin embargo, los dispositivos de monitoreo disponibles en el mercado suelen ser caros y complejos a la hora de usarlos, lo que limita su accesibilidad para muchos pacientes.

En respuesta a esta necesidad, se ha desarrollado una aplicación web accesible y económica para el monitoreo cardíaco en tiempo real. Esta herramienta permite a los usuarios conectar el dispositivo de monitoreo cardíaco a través de un puerto serial, visualizar los datos de ECG en tiempo real y guardar las grabaciones para su posterior análisis, dentro de la misma aplicación web.

Continuando con la iniciativa de proporcionar herramientas de apoyo en el ámbito clínico y de ayuda para los pacientes, la aplicación tiene varias funcionalidades clave:

\begin{itemize}
    \item Conexión y desconexión del dispositivo de monitoreo cardíaco.
    \item Visualización en tiempo real de los datos de ECG.
    \item Cálculo y visualización de la frecuencia cardíaca en LPM.
    \item Grabación y almacenamiento de los datos de ECG.
    \item Predicción y clasificación de latidos cardíacos usando el modelo de predicción Random Forest.
\end{itemize}

Se han añadido mejoras adicionales, como la capacidad de guardar los resultados de las predicciones en una base de datos. Esto se realiza mediante un botón que, una vez generadas las predicciones, permite al usuario guardar estos resultados junto con la fecha y hora en la base de datos, la cual, se puede descargar en un fichero CSV. Una página adicional en la aplicación permite la visualización y gestión de esta base de datos, facilitando el acceso a los resultados históricos tanto para pacientes como para profesionales de la salud.

\end{abstract}

\renewcommand*\abstractname{Descriptores}
\begin{abstract}
Monitoreo cardíaco, ECG, visualización en tiempo real, predicción de latidos cardíacos, Random Forest, base de datos, aplicación web, salud cardíaca, análisis de ECG, accesibilidad tecnológica, innovación en salud.


\end{abstract}

\clearpage

% Abstract en inglés
\renewcommand*\abstractname{Abstract}
\begin{abstract}
 Cardiac activity monitoring is crucial for the early detection and proper management of various cardiac conditions. However, the monitoring devices available on the market are often expensive and complex to use, limiting their accessibility for many patients.

In response to this need, an accessible and economical web application for real-time cardiac monitoring has been developed. This tool allows users to connect the cardiac monitoring device through a serial port, visualize the ECG data in real time, and save the recordings for later analysis within the same web application.

Continuing the initiative to provide support tools in the clinical field and help patients, the application offers several key functionalities:

\begin{itemize}
    \item Connection and disconnection of the cardiac monitoring device.
    \item Real-time visualization of ECG data.
    \item Calculation and visualization of heart rate in BPM.
    \item Recording and storage of ECG data.
    \item Prediction and classification of heartbeats using the Random Forest prediction model.
\end{itemize}

Additional improvements have been added, such as the ability to save the prediction results in a database. This is done through a button that, once the predictions are generated, allows the user to save these results along with the date and time in the data base and download a CSV file. An additional page in the application allows the visualization and management of this database, facilitating access to historical results for both patients and healthcare professionals.
\end{abstract}


\renewcommand*\abstractname{Keywords}
\begin{abstract}
Cardiac monitoring, ECG, real-time visualization, heartbeat prediction, Random Forest, database, web application, cardiac health, ECG analysis, technological accessibility, healthcare innovation.
\end{abstract}

\clearpage

% Indices
\tableofcontents

\clearpage

\listoffigures

\clearpage

\listoftables
\clearpage


\mainmatter
\capitulo{1}{Introducción}


La detección temprana de irregularidades en la actividad cardíaca es clave para la prevención y el tratamiento de enfermedades cardiovasculares. Con la creciente incidencia de enfermedades cardíacas a nivel mundial, surge la necesidad de herramientas accesibles a toda la población que permitan la supervisión constante del estado del corazón. Según la Organización Mundial de la Salud (OMS), las enfermedades cardiovasculares son la principal causa de muerte en el mundo, siendo responsables de aproximadamente 17.9 millones de muertes al año, lo que representa el 31\% de todas las muertes globales \cite{who-cvd}. Este Trabajo de Fin de Grado (TFG) se centra en el desarrollo de un sistema de monitoreo cardíaco en tiempo real junto a una predicción de cada tipo de ciclo cardiaco que nos alerta de irregularidades en él, se utiliza una combinación de hardware y software para ofrecer una solución económica y eficaz.


El proyecto se basa en la utilización de un sensor de ECG (Electrocardiograma) conectado a un microcontrolador Arduino, que envía los datos a una aplicación web para su visualización, análisis y almacenamiento. La implementación de este sistema busca proporcionar a los usuarios una herramienta que no solo monitorice la actividad cardíaca en tiempo real, sino que también grabe los datos para análisis posteriores, indentificando patrones relevantes en cada ciclo cardiaco y asignando una etiqueta a cada segmento.

El sistema incluye una aplicación de escritorio desarrollada con Tkinter, que permite la conexión y visualización de los datos en tiempo real y grabación en un archivo de tipo xls. Además tiene una interfaz web interactiva creada con Streamlit para el análisis de los datos de ECG. La información se almacena en una base de datos SQLite, permitiendo un acceso rápido y eficiente a los datos históricos.

Este documento detalla el proceso de desarrollo del sistema de monitoreo cardíaco, comenzando con la definición de objetivos y fundamentos teóricos, seguido por una descripción de la metodología y las herramientas empleadas. Se incluyen los resultados obtenidos y las conclusiones del trabajo realizado, así como una discusión sobre posibles mejoras y expansiones futuras del proyecto.

El objetivo principal de este proyecto es contribuir a la innovación en el campo de la salud, ofreciendo una solución tecnológica que facilite el monitoreo continuo de la actividad cardíaca, mejorando así la calidad de vida de los pacientes y proporcionando a los profesionales de la salud una herramienta adicional para el diagnóstico y seguimiento de enfermedades cardíacas. En los anexos se proporciona información adicional que podría ser interesante sobre el desarrollo del proyecto.





\capitulo{2}{Objetivos}

Este proyecto tiene como principal objetivo desarrollar un sistema de monitoreo cardíaco en tiempo real que sea accesible, eficaz y económico. Permitirá a los pacientes y a los profesionales de la salud supervisar la actividad cardíaca y detectar de manera temprana cualquier irregularidad. A continuación, se enumeran y explican de forma detallada los objetivos de la realización de este proyecto.

\subsection{Objetivos generales}

El proyecto se centra en la creación de una solución para detectar de manera temprana cualquier irregularidad en el ciclo cardíaco, que combine hardware y software de bajo costo y con una alta eficiencia. Los objetivos generales buscan asegurar que el sistema sea útil y fácil de usar para pacientes, mejorando la calidad de vida y facilitando el diagnóstico y seguimiento de afecciones cardíacas.

\begin{itemize}
\item \textbf{Desarrollar un sistema de monitoreo cardíaco en tiempo real}: Implementar una solución tecnológica que permita la visualización y grabación continua de la actividad cardíaca.
\item \textbf{Mejorar la calidad de vida de los pacientes}: Proporcionar una herramienta de predicción que clasifique cada ciclo cardíaco y permita detectar de manera temprana cualquier irregularidad en la actividad cardíaca.
\item \textbf{Asegurar la accesibilidad y economicidad del sistema}: Utilizar hardware y software de bajo costo y de código abierto para garantizar que la solución sea accesible a una amplia población.
\end{itemize}

\subsection{Objetivos de desarrollo web}

Estos objetivos están relacionados con  la interfaz web con el objetivo de que la experiencia del usuario sea lo mejor posible.

\begin{itemize}
\item \textbf{Crear una interfaz web intuitiva y fácil de usar}: Desarrollar una aplicación web con una interfaz clara y amigable que permita a los usuarios interactuar fácilmente con el sistema.
\item \textbf{Implementar la visualización y grabación de datos en tiempo real}: Mostrar los datos de ECG en una ventana con gráficos interactivos que se actualicen en tiempo real, en la cual se pueda grabar los datos de ECG en un fichero para que los usuarios guarden sus datos de ECG para un análisis posterior mediante la predicción de cada ciclo cardíaco.
\item \textbf{Proporcionar herramientas básicas}: Ofrecer herramientas para el análisis de los datos almacenados como puede ser, el seleccionar el rango de datos (en tiempo) sobre el que quieras realizar la predicción. Además de herramientas para realizar cambios en la base de datos donde se almacenan todas las predicciones y diferentes botones para guardar los datos o subir los archivos para su análisis.   
\end{itemize}

\subsection{Objetivos de Integración y Funcionalidad del Sistema}

Los objetivos en esta sección se enfocan en garantizar la integración efectiva del hardware con el software, asegurando que todas las partes del sistema funcionen en conjunto de manera eficiente y confiable.

\begin{itemize}
\item \textbf{Integrar el hardware con la aplicación web}: Asegurar que los datos recogidos por el hardware se transmitan correctamente a la aplicación web.
\item \textbf{Desarrollar la funcionalidad de conexión y desconexión del dispositivo de monitoreo}: Permitir a los usuarios conectar y desconectar el dispositivo de monitoreo cardíaco a través de la interfaz web.
\item \textbf{Implementar una base de datos robusta}: Utilizar SQLite para almacenar de manera eficiente las predicciones.
\item \textbf{Crear ventanas interactivas para la selección y análisis de datos}: Permitir a los usuarios seleccionar ventanas de tiempo específicas para analizar los datos de ECG.
\item \textbf{Desarrollar la funcionalidad de predicción utilizando machine learning}: Implementar modelos de machine learning, como Random Forest, para realizar predicciones sobre los datos de ECG.
\item \textbf{Optimizar el modelo de predicción de Random Forest}: Ajustar los parámetros del modelo para mejorar la precisión en la clasificación de los diferentes tipos de latidos cardíacos que hay en la muestra.
\end{itemize}

\subsection{Objetivos de Desarrollo Hardware}

Estos objetivos se centran en la mejora y optimización del hardware utilizado para el monitoreo cardíaco, asegurando su eficiencia y facilidad de uso, es decir, que sea lo más automático posible.

\begin{itemize}
\item \textbf{Optimizar el diseño del hardware}: Mejorar el diseño del sistema de monitoreo cardíaco para un mejor y más sencillo manejo. Esto implica incluir en el código de Arduino varias funcionalidades clave:
\begin{enumerate}
    \item \textbf{Calibración automática}: Durante el tiempo de calibración inicial, el sistema ajusta el umbral de detección de latidos basándose en los valores máximos y mínimos de la señal recibida. Esto asegura que el sistema sea adaptable a diferentes usuarios y condiciones.
    \item \textbf{Detección de latidos}: Utiliza el valor umbral calibrado para detectar picos en la señal de ECG que corresponden a los latidos cardíacos. Cuando un latido es detectado, se enciende un LED para proporcionar una señal visual.
    \item \textbf{Filtrado}: Filtra y procesa la señal del ECG para mejorar la precisión de la detección de latidos, minimizando el impacto del ruido y otras interferencias.
\end{enumerate}


\item \textbf{Implementar un LED que se encienda con el pulso del paciente}: Añadir un LED que indique visualmente el ritmo cardíaco del paciente. Este LED servirá también como una confirmación visual del correcto funcionamiento del dispositivo.
\item \textbf{Desarrollar una caja protectora impresa en 3D}: Crear una caja mediante impresión 3D para alojar y proteger el microcontrolador Arduino y los componentes electrónicos. La caja estará diseñada para ofrecer un acceso fácil a los puertos y conexiones necesarias, al tiempo que proporciona protección contra el polvo y posibles impactos.

\end{itemize}

Estos objetivos buscarán asegurar que todo sea fácil de usar y cómodo para los pacientes.

\capitulo{3}{Conceptos teóricos}

\section{Conceptos teóricos básicos}

En este capítulo se presentan los conceptos teóricos fundamentales que sustentan el desarrollo del sistema de monitoreo cardíaco en tiempo real, permitiendo a cualquier lector comprender el trabajo realizado. Se abordarán temas relacionados con las enfermedades cardiovasculares, la tecnología utilizada, y los métodos de análisis de datos.

\subsection{El Sistema cardiovascular}

El sistema cardiovascular está compuesto por el corazón, los vasos sanguíneos y la sangre. Su función principal es el transporte de oxígeno y nutrientes a los tejidos del cuerpo y la eliminación de desechos metabólicos. El corazón actúa como una bomba que impulsa la sangre a través de los vasos sanguíneos, y la función adecuada de este sistema es crucial para mantener la homeostasis y la salud general del cuerpo \cite{guyton2006text}.

\subsection{Anatomía del Sistema Cardiovascular}

\subsubsection{El corazón}
El corazón es un órgano muscular situado en el mediastino, entre los pulmones. Está compuesto por cuatro cámaras: dos aurículas (derecha e izquierda) y dos ventrículos (derecho e izquierdo). La sangre fluye a través de estas cámaras, siendo impulsada por las contracciones rítmicas del músculo cardíaco. Las paredes del corazón están formadas por tres capas: el endocardio, el miocardio y el pericardio \cite{sistema_cardiovascular}.

\subsubsection{Los vasos sanguíneos}
Existen tres tipos principales de vasos sanguíneos: arterias, venas y capilares. Las arterias transportan la sangre rica en oxígeno desde el corazón hacia los tejidos del cuerpo, mientras que las venas devuelven la sangre pobre en oxígeno al corazón. Los capilares, que son los vasos más pequeños, facilitan el intercambio de oxígeno, nutrientes y desechos entre la sangre y los tejidos \cite{sistema_cardiovascular}.

\subsection{Función del sistema cardiovascular}

El sistema cardiovascular realiza varias funciones esenciales:
\begin{itemize}
    \item \textbf{Transporte de nutrientes y oxígeno:} La sangre transporta oxígeno y nutrientes a las células del cuerpo y elimina dióxido de carbono y otros desechos metabólicos \cite{sistema_cardiovascular}.
    \item \textbf{Regulación de la temperatura corporal:} Mediante la redistribución del flujo sanguíneo, el sistema cardiovascular ayuda a mantener la temperatura corporal \cite{sistema_cardiovascular}.
    \item \textbf{Protección:} La sangre contiene células inmunitarias que defienden al cuerpo contra infecciones y enfermedades \cite{guyton2006text}.
    \item \textbf{Homeostasis:} Mantiene el equilibrio de electrolitos y el pH en el cuerpo, cruciales para el funcionamiento normal de las células \cite{guyton2006text}.
\end{itemize}

\subsection{Enfermedades cardiovasculares}

Las enfermedades cardiovasculares (ECV) son la principal causa de muerte a nivel mundial, representando aproximadamente el 31\% de todas las muertes globales según la Organización Mundial de la Salud (OMS) \cite{who-cvd}. Estas enfermedades incluyen trastornos como la cardiopatía coronaria, la insuficiencia cardíaca, la hipertensión arterial y las arritmias, entre otros. La detección temprana y el monitoreo continuo son cruciales para la prevención y el tratamiento efectivo de estas patologías.

\subsection{Tipos de enfermedades cardiovasculares}

Las enfermedades cardiovasculares corresponden a los trastornos del sistema circulatorio, que incluye el corazón, los vasos sanguíneos y la sangre. Se clasifican en cuatro tipos generales: enfermedades isquémicas del corazón, enfermedades cerebrovasculares, enfermedades vasculares periféricas y otras enfermedades \cite{corella2007enfermedades}.

\subsubsection{Enfermedades isquémicas del corazón}

Estas enfermedades se deben a un estrechamiento progresivo de las arterias coronarias causado por la formación de placas de ateroma. Este engrosamiento de la pared arterial obstruye el flujo sanguíneo, lo que puede provocar isquemia y, si persiste, infarto de miocardio \cite{corella2007enfermedades}.

\subsubsection{Enfermedades cerebrovasculares}

Estas enfermedades se deben a alteraciones en la circulación cerebral. Se clasifican en isquémicas y hemorrágicas. Las isquémicas se producen por una disminución del flujo sanguíneo hacia una región del cerebro, causando infarto cerebral. Las hemorrágicas se deben a la rotura de un vaso sanguíneo en el cerebro \cite{corella2007enfermedades}.

\subsubsection{Enfermedades vasculares periféricas}

Afectan a las arterias o venas que irrigan las extremidades. Provocan dificultades en la circulación sanguínea, estrechamiento de los vasos, hinchazón y dolor. Pueden causar isquemia y, en el caso de las venas, trombosis venosa \cite{corella2007enfermedades}.

\subsubsection{Otras enfermedades cardiovasculares}

Incluyen las cardiopatías congénitas y la cardiopatía reumática. La cardiopatía reumática se produce por infecciones bacterianas que causan lesiones en el miocardio y en las válvulas del corazón \cite{corella2007enfermedades}.

\subsection{Introducción a la monitorización cardíaca}

La monitorización cardíaca es una técnica utilizada para observar y registrar la actividad eléctrica del corazón a lo largo del tiempo. Esta práctica es esencial para la detección de diversas condiciones cardíacas, incluyendo arritmias, infartos de miocardio y otras patologías cardíacas. Los dispositivos de monitorización cardíaca pueden ser no invasivos, como los electrocardiogramas (ECG) tradicionales y los monitores Holter, o invasivos, como los dispositivos implantables que registran la actividad cardíaca de forma continua \cite{MayoClinic_monitoring}.

La importancia de la monitorización cardíaca radica en su capacidad para proporcionar datos críticos que ayudan en el diagnóstico temprano, el manejo y el tratamiento de enfermedades cardíacas. Estos datos permiten a los profesionales de la salud tomar decisiones informadas y personalizar los tratamientos para mejorar los resultados de los pacientes \cite{MayoClinic_monitoring}.

\subsection{Electrocardiograma (ECG)}

El electrocardiograma (ECG) es una herramienta diagnóstica utilizada en los hospitales y centros de salud principalmente, que registra la actividad eléctrica del corazón. Con la función de detectar diversas anomalías cardíacas, como por ejemplo arritmias, infartos de miocardio y otros trastornos del ritmo cardíaco. Un ECG típico incluye las ondas P, Q, R, S y T, que representan diferentes fases del ciclo cardíaco. Toda la información a continuación se ha extraído de \cite{azcona2009electrocardiograma}.

\subsubsection{Definición de un electrocardiograma}

El ECG se trata de un estudio de las variaciones del voltaje en relación con el tiempo, esto se ve representado en una gráfica. Se mide la corriente eléctrica que se está desarrollando en el corazón durante un tiempo determinado, en un ECG normal no suele exceder los 30 segundos la duración de la medición.

\subsubsection{El sistema de conducción del corazón}

Para entender bien las razones del por qué y cómo oscilan las líneas del ECG, hay que conocer y entender el sistema de conducción eléctrica del corazón, el cual está formado por el nodo sinoauricular (nodo SA), el nodo auriculoventricular (nodo AV) y el sistema de His-Purkinje, como se puede ver en la Figura (\ref{fig:conduccion_cardiaca}) \cite{enfermeriatop2024}.

\begin{figure}[h]
    \centering
    \includegraphics[width=0.6\textwidth]{img/conduccion_cardiaca.png}
    \caption{Sistema de conducción del corazón \cite{enfermeriatop2024}.}
    \label{fig:conduccion_cardiaca}
\end{figure}

\paragraph{Nodo sinoauricular (SA)}
Esta estructura tiene una forma de semiluna y se encuentra localizada detrás de la aurícula derecha, es el lugar donde se inicia el impulso eléctrico del corazón.

\paragraph{Nodo auriculoventricular (AV)}
Se encuentra en la unión entre aurículas y ventrículos, en cuanto al tamaño es la mitad que el del nodo SA. En este caso, el nodo AV tiene como función que las aurículas se contraigan y vacíen su contenido de sangre en los ventrículos antes de producirse la contracción ventricular.

\paragraph{Sistema de His-Purkinje}
El impulso cardíaco se propaga por el haz de His y sus ramas a lo largo del tabique interventricular y se distribuye por toda la masa ventricular a través de las fibras de Purkinje, provocando la contracción de los ventrículos.

\subsection{Interpretación de un electrocardiograma}

El ECG presenta como línea guía la denominada línea isoeléctrica o línea basal. Los latidos cardíacos quedan representados en el ECG por las diferentes oscilaciones de la línea basal en forma de ángulos, segmentos, ondas e intervalos.

\begin{figure}[h]
    \centering
    \includegraphics[width=0.5\textwidth]{img/ecg_wave.png}
    \caption{Onda típica del ECG \cite{wikipedia-ecg}.}
    \label{fig:ecg_wave}
\end{figure}

\subsubsection{Onda P}
Es la primera elevación de la onda que aparece en el ECG. Es el momento en que las aurículas se están contrayendo y enviando sangre hacia los ventrículos. Suele durar unos 0.08 segundos.

\subsubsection{Segmento PR}
Es el tramo que aparece entre el final de la onda P y el inicio del complejo QRS. Es el tiempo que tarda el impulso eléctrico en pasar a través del nodo AV.

\subsubsection{Intervalo PR}
El intervalo PR es el período desde el inicio de la onda P hasta el inicio del complejo QRS. Es el tiempo total de conducción auriculoventricular, es decir, el tiempo que tarda el impulso eléctrico en viajar desde el nodo SA, pasando por el nodo AV, hasta el inicio de la despolarización ventricular. Suele durar entre 0.12 y 0.20 segundos \cite{azcona2009electrocardiograma}.

\subsubsection{Complejo QRS}
Es el momento en que los ventrículos se contraen y expulsan su contenido sanguíneo. Consta de las ondas Q, R y S. El complejo QRS no debe exceder en duración más de 0.08 segundos.

\subsubsection{Segmento ST}
Es el trazado de la línea basal entre el final de la onda S y el comienzo de la onda T. Dependiendo de que sufra una elevación o descenso puede indicar insuficiencia en el riego del corazón.

\subsubsection{Onda T}
Sucede después del segmento ST. Es el momento en que el corazón se encuentra en un período de relajación, una vez que ha expulsado la sangre de los ventrículos.

\subsubsection{Intervalo QT}
El intervalo QT se extiende desde el inicio del complejo QRS hasta el final de la onda T. Representa el tiempo total que tarda el músculo ventricular en despolarizarse y repolarizarse. La duración del intervalo QT varía con la frecuencia cardíaca, pero generalmente dura entre 0.36 y 0.44 segundos \cite{azcona2009electrocardiograma}.

\subsection{Importancia de un electrocardiograma}

Realizar un ECG es un procedimiento sencillo que requiere de un electrocardiógrafo, parches de ECG y un sistema de cables. El paciente se coloca boca arriba y se le colocan los parches, cuatro en las extremidades y seis en puntos específicos del pecho, formando así 12 derivaciones.

Los objetivos del ECG principalmente para detectar trastornos del ritmo cardíaco (arritmias) y en el diagnóstico de situaciones con aporte insuficiente de sangre al corazón (infarto de miocardio y angina de pecho).

La importancia del monitoreo continuo reside en la detección temprana de enfermedades cardiovasculares y otras anomalías del ritmo cardíaco, como arritmias. La electrocardiografía (ECG) es una técnica ampliamente utilizada para registrar la actividad eléctrica del corazón. Los datos de ECG permiten a los médicos diagnosticar diversas afecciones cardíacas, desde arritmias hasta enfermedades coronarias. Un ECG típico mide la actividad eléctrica en diferentes puntos del tiempo, proporcionando información sobre el ritmo y la fuerza de los latidos del corazón \cite{MayoClinic_monitoring}.

\subsection{Tecnologías de comunicación}

El sistema de monitoreo utiliza una conexión serial a través del puerto USB para transmitir datos del Arduino a la aplicación web. Este método de comunicación es sencillo y efectivo, lo ideal sería conexión inalámbrica mediante Wi-Fi o Bluetooth pero mirando el lado positivo, realizando la conexión mediante USB para la transmisión de los datos se está eliminando la necesidad de baterías y otra serie de problemas que aparecen en módulos inalámbricos como el HC-05.

\subsection{Comunicación serial}

La comunicación serial permite la transmisión de datos entre el Arduino y la aplicación web a través de un cable USB. Es una forma confiable y directa de transferir grandes volúmenes de datos en tiempo real.

La comunicación serial en Arduino es fundamental para la interacción entre el microcontrolador y dispositivos externos. Utilizando esta comunicación, el Arduino puede enviar y recibir datos en tiempo real, lo que es ideal para proyectos que requieren interacción con sensores, displays u otros componentes de hardware \cite{TechZero2024}.

La configuración de la comunicación serial comienza con la inicialización de la misma mediante la función \texttt{Serial.begin(baudRate)} en el sketch de Arduino. El parámetro \texttt{baudRate} establece la velocidad de transmisión de datos, que debe coincidir en ambos extremos de la comunicación para evitar errores \cite{DeepBluEmbedded2024}.

La comunicación serial es eficiente y ligera en comparación con otras formas de transmisión de datos, proporcionando un control preciso sobre el flujo de información y asegurando una transmisión precisa sin demoras ni errores significativos \cite{TechZero2024}.

Para proyectos que requieren el envío de datos en tiempo real a una aplicación web, la comunicación serial a través de USB ofrece una solución simple y efectiva, facilitando la creación de sistemas de monitoreo interactivos y de alta fiabilidad.

\subsection{Sensor AD8232 y su funcionamiento}

El sensor AD8232 es un módulo especializado en la captación de señales cardíacas, comúnmente utilizado para la creación de electrocardiogramas (ECG). Este sensor es altamente eficiente para detectar la actividad eléctrica del corazón, proporcionando una solución precisa y económica para proyectos de monitoreo cardíaco.

\textbf{Características del sensor AD8232}

El AD8232 destaca por varias características clave que lo hacen ideal para aplicaciones de monitoreo de salud. Entre ellas se encuentran su alta precisión para filtrar y amplificar las señales bioeléctricas del corazón, eliminando el ruido y las interferencias \cite{AD8232_teoria}. Además, el sensor tiene un bajo consumo de energía, lo que lo hace perfecto para dispositivos portátiles \cite{SparkFun_AD8232}. También es fácil de integrar con diversas plataformas de microcontroladores, como Arduino, facilitando su implementación en proyectos de monitoreo cardíaco \cite{liaw2002classification}.

\textbf{Funcionamiento del sensor AD8232}

El sensor AD8232 capta las señales eléctricas del corazón a través de electrodos colocados en la piel del paciente. Estas señales son amplificadas y filtradas para eliminar el ruido y las interferencias, proporcionando una salida analógica que puede ser leída y procesada por un microcontrolador, como una placa Arduino. La señal resultante es una representación precisa de la actividad eléctrica del corazón, similar a un ECG, que puede ser utilizada para monitorear y analizar el ritmo cardíaco en tiempo real \cite{Kumar2020_LowCostECG}.

\textbf{Descripción detallada del sensor AD8232}

El sensor AD8232 incluye varios componentes principales: los electrodos, el amplificador AD8232, los filtros y las salidas. Los electrodos detectan la actividad eléctrica del corazón, que es amplificada por el AD8232. Los filtros incorporados eliminan el ruido, dejando una señal clara del ECG que puede ser visualizada y analizada en tiempo real \cite{AD8232_teoria}.

\textbf{Proceso de medición}

El proceso de medición con el AD8232 incluye la correcta colocación de los electrodos en el cuerpo: RA (Brazo Derecho), LA (Brazo Izquierdo) y RL (Pierna Derecha). Los electrodos detectan las señales eléctricas generadas por la actividad del corazón, que son amplificadas y filtradas por el AD8232. La señal amplificada se envía a un microcontrolador o dispositivo de grabación a través de la salida analógica, donde puede ser visualizada y analizada en tiempo real \cite{SparkFun_AD8232}.

\textbf{Conexiones del sensor AD8232}

El AD8232 tiene varias conexiones importantes: la salida analógica (OUT) se conecta a una entrada analógica de un microcontrolador, los pines de alimentación (3.3V/5V y GND) se conectan a una fuente de alimentación adecuada, y los pines de electrodo (RA, LA, RL) se conectan a los respectivos electrodos en el cuerpo \cite{AD8232_teoria}.


\subsection{Desarrollo web}

Para el desarrollo de la aplicación web se usan las tecnologías frontend y backend que colaboran para darnos una interfaz intuitiva y funcional. Es clave entender las diferencias y las interacciones entre el frontend y el backend para crear un software efectivo \cite{PrimeIT}.

\subsubsection{Frontend}

El frontend de la aplicación web ha sido desarrollado con el framework Streamlit, que permite la creación rápida de interfaces interactivas en Python. Streamlit facilita la visualización de datos en tiempo real y proporciona una experiencia de usuario fluida. El frontend es el diseño y desarrollo de la interfaz de usuario, aspectos como la accesibilidad, la estructura de navegación, la capacidad de respuesta y las animaciones. Se utilizan lenguajes como CSS, HTML y JavaScript \cite{PrimeIT}. Esta enfocado en la experiencia del usuario, asegurando que la interfaz sea intuitiva y atractiva.

\subsubsection{Backend}

El backend, también desarrollado en Python, se encarga del procesamiento de datos y la gestión de la base de datos SQLite. Esta base de datos almacena las predicciones realizadas por el modelo de machine learning. El backend es la parte de la funcionalidad, que se encarga de gestionar los recursos, implementando la lógica del sistema e integrando los servidores web y bases de datos. Entre los lenguajes de programación comúnmente utilizados en el backend se incluyen Java, Python y PHP \cite{PrimeIT}.

\subsection{Desarrollo de la base de datos}

Para almacenar todos los datos obtenidos durante la predicción de clases de ciclos cardiacos del paciente, es necesario diseñar una base de datos adecuada. El diseño de bases de datos implica decidir su propósito y organizar la información por temas. Cada tabla debe tener claves únicas que identifiquen cada fila de manera individual y definir cómo se relacionan entre sí. Existen dos tipos principales de bases de datos:


\subsubsection{Bases de datos relacionales}

Este tipo de base de datos se busca mantener la máxima integridad y precisión de la información, lo cual puede causar una mayor lentitud por que se realiza el procesamiento de los datos antes que el almacenamiento. Está diseñada para garantizar la exactitud y consistencia de los datos, priorizando la eficiencia y veracidad sobre la velocidad. Ejemplos de bases de datos relacionales son PostgreSQL y MySQL \cite{Structuralia}. En este proyecto, se ha utilizado SQLite, una base de datos relacional.

\subsubsection{Bases de datos no relacionales}

Por otro lado, las bases de datos no relacionales buscan el almacenamiento rápido de los datos sobre la exactitud, reduciendo las restricciones y, a veces, sacrificando la veracidad. El análisis de los datos se realiza posteriormente. Este tipo de base de datos es adecuado cuando la velocidad es crucial y la estructura de los datos puede ser más flexible. Firebase es un ejemplo de una base de datos no relacional \cite{Structuralia}.

\subsection{Visualización y grabación de datos}

El sistema cuenta con una aplicación desarrollada en Tkinter, que permite la visualización y grabación de los datos de ECG en tiempo real. Tkinter es una biblioteca de interfaces gráficas para Python que facilita la creación de ventanas y elementos interactivos \cite{tkinter_docs}.

Para la visualización de datos en tiempo real, Tkinter utiliza su función de actualización periódica que permite refrescar la interfaz con nuevos datos continuamente. Esto es esencial en aplicaciones de monitoreo como el ECG, donde es crucial visualizar las señales cardíacas a medida que se generan \cite{tkinter_real_time}.

La integración de Tkinter con bibliotecas de visualización de gráficos como Matplotlib permite representar los datos de ECG en gráficos dinámicos. Matplotlib facilita la creación de gráficos en 2D que se pueden integrar fácilmente en las ventanas de Tkinter, proporcionando una visualización clara y en tiempo real de los datos de ECG \cite{matplotlib}.

La grabación de datos en Tkinter se puede manejar mediante la captura de datos en intervalos regulares y su almacenamiento en archivos o bases de datos. Esto es útil para realizar análisis posteriores o para mantener un registro histórico de las señales de ECG para su revisión \cite{tkinter_data_logging}.

En resumen, la combinación de Tkinter y Matplotlib proporciona una plataforma poderosa y flexible para la visualización y grabación de datos en tiempo real en aplicaciones médicas, como el monitoreo cardíaco, ofreciendo una interfaz intuitiva y funcional para los usuarios \cite{tkinter_matplotlib_integration}.

\subsection{Modelo de predicción}

Se ha utilizado un modelo de Random Forest, este modelo ha sido entrenado con datos de ECG obtenidos de Kaggle \cite{kaggle-data}, que incluyen etiquetas como Latidos Normales, Latidos de Ectopia Supraventricular, Latidos de Ectopia Ventricular, Latidos de Fusión y Latidos Inclasificables.

El modelo de predicción basado en Random Forest es una técnica de aprendizaje automático utilizada tanto para problemas de clasificación como de regresión. Es conocido por su alta precisión, capacidad de manejar grandes conjuntos de datos y su resistencia al sobreajuste (overfitting). A continuación, se presenta una explicación teórica detallada del modelo Random Forest, junto con referencias bibliográficas para una comprensión más profunda \cite{randomforest-medium}.

El Random Forest es un conjunto de árboles de decisión entrenados de manera individual en diferentes subconjuntos del conjunto de datos. Cada árbol vota por una clase y la clase con más votos se selecciona como la predicción final. Este enfoque reduce el riesgo de sobreajuste y mejora la precisión del modelo al promediar los resultados de múltiples árboles. Random Forest también proporciona una medida de importancia de las características, lo que puede ser útil para entender cuáles son las características más relevantes para la predicción \cite{randomforest-medium}.

La Figura \ref{fig:RandomForest} muestra un diagrama del modelo Random Forest utilizado para la predicción de latidos cardíacos.

\begin{figure}[h]
\centering
\includegraphics[width=0.5\textwidth]{img/randomforest.png}
\caption{Diagrama del modelo Random Forest utilizado para la predicción de latidos cardíacos \cite{randomforest-medium}.}
\label{fig:RandomForest}
\end{figure}

\subsection{Fundamentos del random forest}

Random Forest es un algoritmo de aprendizaje de conjunto (ensemble learning) que combina múltiples árboles de decisión para mejorar la precisión y robustez del modelo. Un árbol de decisión es una estructura de datos en forma de árbol donde cada nodo interno representa una "prueba" en una característica, cada rama representa el resultado de la prueba, y cada hoja representa una etiqueta de clase (en clasificación) o un valor continuo (en regresión) \cite{breiman2001random}. Random Forest utiliza el método de Bootstrap Aggregating (Bagging), que genera múltiples subconjuntos del conjunto de datos original mediante muestreo con reemplazo. Cada subconjunto se utiliza para entrenar un árbol de decisión independiente, reduciendo la varianza del modelo y mejorando la estabilidad y precisión \cite{breiman1996bagging}. Además, para cada nodo en un árbol de decisión, Random Forest selecciona aleatoriamente un subconjunto de características y elige la mejor división basada en estas características, lo cual reduce la correlación entre los árboles individuales y mejora la precisión del modelo. Una vez que todos los árboles han sido entrenados, el resultado final de Random Forest se obtiene mediante la agregación de las predicciones de todos los árboles. En problemas de clasificación, se utiliza la votación mayoritaria, mientras que en problemas de regresión se calcula el promedio de las predicciones \cite{breiman2001random}.

Se emplea un modelo de machine learning basado en Random Forest para realizar predicciones sobre los segmentos de ECG. Este modelo ha sido entrenado utilizando datos de Kaggle \cite{kaggle-data}, y es capaz de clasificar diferentes tipos de latidos cardíacos.

\subsection{Preparación de los datos}

Los datos de ECG se preprocesaron para extraer características relevantes que el modelo pudiera utilizar para la predicción. Este proceso incluyó la normalización de los datos y la segmentación de los ciclos cardíacos. Cada segmento del ECG se etiquetó con el tipo de latido correspondiente.

\subsubsection{Detección de picos R}

La detección de picos R es crucial para identificar los latidos del corazón en los datos de ECG. Se utiliza la función \texttt{find\_peaks} de Scipy para localizar estos picos.

\subsubsection{Segmentación de latidos}

Los segmentos de latidos se extraen centrados en los picos R detectados. Cada segmento incluye una cantidad fija de muestras antes y después del pico R para capturar un ciclo cardíaco completo y se rellena con valores 0 hasta que alcanza las 187 muestras, esto ocurre porque tiene que ser de la misma longitud que los datos de entrenamiento y test, los cuales cada segmento que corresponde a un latido está conformado por 187 muestras.

\subsection{Entrenamiento del modelo}

El modelo de Random Forest se entrenó utilizando el conjunto de datos preprocesados. Se utilizó una validación cruzada para evaluar el rendimiento del modelo y ajustar los parámetros para mejorar su precisión. El modelo final se guardó utilizando la biblioteca \texttt{Joblib} para su posterior uso en la aplicación web.

\subsection{Implementación en la aplicación web}

El modelo de predicción se integró en el backend de la aplicación web desarrollada con Streamlit. Cuando los datos de ECG se reciben en la aplicación, se procesan y segmentan en tiempo real. Cada segmento se pasa al modelo de Random Forest, que predice el tipo de latido. Los resultados se muestran en la interfaz de usuario y se pueden almacenar en la base de datos para su posterior análisis.

\subsection{Elección del modelo predictivo: Random Forest}

La elección del modelo de predicción es una decisión muy importante en el desarrollo de cualquier sistema de aprendizaje automático. En este proyecto, se ha optado por utilizar el modelo Random Forest para la clasificación de latidos cardíacos por las razones que se van a explicar y demostrar a continuación.

\subsubsection{Ventajas del Random Forest}

El modelo Random Forest se ha elegido por varias razones clave, que se detallan a continuación:

\begin{itemize}
\item \textbf{Robustez y estabilidad:} Random Forest es un modelo de ensamblaje que combina múltiples árboles de decisión, lo que reduce la varianza del modelo y mejora la precisión al evitar el sobreajuste. Evitar el sobreajuste es algo clave de cara a nuestro proyecto ya que va a trabajar todo el rato con datos nuevos y puede que de pacientes diferentes, por lo tanto es esencial para asegurar que el modelo funcione bien con datos nuevos y no vistos \cite{breiman2001random}.
\item \textbf{Manejo de datos faltantes y no lineales:} Una de las principales ventajas de Random Forest es su capacidad para manejar datos faltantes y capturar relaciones no lineales en los datos. Esto es especialmente útil en aplicaciones médicas donde los datos pueden estar incompletos o presentar relaciones complejas \cite{ho1995random}.
\item \textbf{Facilidad de uso e implementación:} Random Forest requiere menos ajuste de parámetros en comparación con otros modelos complejos como las redes neuronales. Además, su implementación es sencilla utilizando bibliotecas como scikit-learn en Python, lo que facilita su uso en diferentes aplicaciones \cite{scikit-learn}.
\item \textbf{Interpretabilidad:} A diferencia de otros modelos de caja negra, Random Forest permite a los desarrolladores entender qué características son más influyentes en las predicciones \cite{liaw2002classification}.
\end{itemize}

\subsubsection{Comparación con otros modelos}

Aunque otros modelos como las redes neuronales artificiales (ANN), K-Nearest Neighbors (KNN), Support Vector Machine (SVM), Gradient Boosting y Decision Trees también son efectivos, Random Forest ofrece una combinación única de características que lo hacen ideal para este proyecto y a continuación se va a realizar una comparación con otros modelos usando en todos ellos los mismos conjuntos de datos de entrenamiento y de test, los resultados obtenidos aparecen en la Tabla \ref{tab:comparison}.

\begin{itemize}
\item \textbf{ANN:} Las ANN pueden ofrecer un rendimiento ligeramente superior en términos de métricas, pero requieren más recursos computacionales y tiempo para el entrenamiento. Además, las ANN son menos interpretables que Random Forest \cite{goodfellow2016deep}.
\item \textbf{KNN y SVM:} Aunque son efectivos, estos modelos pueden ser menos escalables y más sensibles a los datos ruidosos en comparación con Random Forest \cite{hastie2009elements}.
\item \textbf{Gradient boosting:} Ofrece un alto rendimiento pero es más complejo de ajustar y entrenar en comparación con Random Forest \cite{friedman2001greedy}.
\item \textbf{Decision tree:} Aunque es sencillo e interpretable, un solo árbol de decisión no es tan robusto ni preciso como un conjunto de árboles como Random Forest \cite{breiman2001random}.
\end{itemize}

\subsubsection{Resultados del modelo Random Forest}

Los resultados obtenidos con el modelo de Random Forest se detallan en la siguiente tabla \ref{tab:comparison}:

\begin{table}[h]
    \centering
    \begin{tabular}{lcccc}
        \toprule
        Modelo & Precisión & F1 Score & Precisión & Recall \\
        \midrule
        Random Forest & 0.9747 & 0.9731 & 0.9748 & 0.9747 \\
        Decision Tree & 0.9527 & 0.9528 & 0.9530 & 0.9527 \\
        SVM & 0.9680 & 0.9657 & 0.9676 & 0.9680 \\
        KNN & 0.9736 & 0.9725 & 0.9727 & 0.9736 \\
        Gradient Boosting & 0.9644 & 0.9621 & 0.9634 & 0.9644 \\
        ANN & 0.9756 & 0.9750 & 0.9751 & 0.9756 \\
        \bottomrule
    \end{tabular}
    \caption{Comparación de métricas de diferentes modelos predictivos con el conjunto de datos de entrenamiento con el mismo conjunto de datos de test para cada modelo}
    \label{tab:comparison}
\end{table}

\subsubsection{Conclusión}

En conclusión, Random Forest se destaca por su combinación de robustez, interpretabilidad, facilidad de uso y capacidad para manejar grandes conjuntos de datos de manera eficiente. Estas características lo hacen especialmente adecuado para la tarea de clasificación de latidos cardíacos en este proyecto, proporcionando un equilibrio óptimo entre rendimiento y practicidad.

\section{Estado del arte y trabajos relacionados}

Para entender mejor la importancia de este proyecto en el campo de las tecnologías médicas, es importante ver qué se ha hecho antes y qué avances tecnológicos existen. Esta revisión bibliográfica examina los desarrollos más recientes y analiza proyectos e investigaciones similares. El objetivo es entender la relevancia y el posible impacto del proyecto, proporcionando una base sólida para su desarrollo y asegurando que esté alineado con las necesidades y desafíos actuales del sector.


\subsection{Revisión de tecnologías}

En los últimos años, las tecnologías de monitoreo cardíaco han avanzado significativamente, permitiendo un monitoreo más preciso y continuo de la actividad cardíaca. A continuación, se revisan algunas de las tecnologías más relevantes:

\subsubsection{Sensores de ECG}

Los sensores de ECG son fundamentales para el monitoreo de la actividad eléctrica del corazón. Estos sensores capturan señales bioeléctricas del corazón, que luego son procesadas y analizadas para detectar irregularidades. Los sensores más avanzados utilizan tecnologías de alto rendimiento para garantizar lecturas precisas y consistentes, incluso en entornos móviles.

\subsubsection{Algoritmos de detección de anomalías}

Los algoritmos de machine learning, como Random Forest, se utilizan para detectar anomalías en los datos de ECG. Estos algoritmos analizan grandes volúmenes de datos y aprenden a identificar patrones asociados con condiciones cardíacas específicas. La implementación de estos algoritmos en dispositivos portátiles mejora la capacidad de monitoreo en tiempo real y la precisión en la detección de eventos cardíacos críticos \cite{breiman2001random}.

\subsubsection{Comunicación inalámbrica}

La comunicación inalámbrica, incluyendo tecnologías como Bluetooth Low Energy (BLE) y Wi-Fi, permite la transmisión de datos de ECG desde dispositivos portátiles a aplicaciones móviles y plataformas web. Esto facilita el monitoreo remoto y continuo de los pacientes, mejorando la accesibilidad y la capacidad de respuesta de los profesionales de la salud \cite{TechZero2024}.

\subsection{Revisión de proyectos similares}


Examinar trabajos anteriores en este campo es clave para poder entender las tendencias y retos actuales. Este estudio nos ayuda a detectar qué falta o necesita mejorar en las investigaciones previas. Al conocer mejor las necesidades de los usuarios, podemos dirigir el nuevo proyecto para que sea más útil y específico.

KardiaMobile es un dispositivo portátil de ECG que permite a los usuarios grabar un electrocardiograma en cualquier lugar y momento. Se conecta a una aplicación móvil, ofreciendo resultados instantáneos y la capacidad de enviar los datos directamente a un profesional de la salud para su análisis. Este dispositivo destaca por su facilidad de uso y portabilidad, lo que facilita el monitoreo continuo de los pacientes \cite{kardiamobile}.

El CheckMe Lite no solo realiza lecturas de ECG, sino que también mide la saturación de oxígeno en sangre (SpO2) y la frecuencia cardíaca. Este dispositivo multifuncional proporciona una amplia gama de datos clínicos, siendo ideal para el monitoreo de la salud en general. Su capacidad para integrar múltiples funciones en un solo dispositivo lo convierte en una herramienta valiosa para el monitoreo en el hogar \cite{checkme}.

El Wellue ER1 es un monitor de ECG portátil que permite realizar un seguimiento de la salud cardíaca a largo plazo. Compatible con smartphones, facilita el almacenamiento y la revisión de datos de ECG a través de aplicaciones móviles, ofreciendo a los usuarios una herramienta poderosa para el monitoreo continuo de su salud cardíaca. Su diseño portátil y su facilidad de uso lo hacen accesible para una amplia gama de usuarios \cite{wellue}.

\subsection{Conclusión}

La revisión del estado del arte y los trabajos relacionados muestra que, aunque existen diversas aplicaciones y dispositivos de monitoreo cardíaco, no todos nos dan una solución integrada de bajo costo y fácil acceso para el monitoreo en tiempo real. Este proyecto se alinea con las necesidades actuales del sector médico al proporcionar una herramienta accesible y efectiva para el monitoreo cardíaco continuo, lo que puede tener un impacto significativo en la prevención y tratamiento de enfermedades cardiovasculares.

































\capitulo{4}{Metodología}

\section{Descripción de los datos}
Este proyecto se basa en el desarrollo de un sistema de monitoreo cardíaco que permite la supervisión continua y en tiempo real de la actividad cardíaca a través de un dispositivo de monitoreo conectado vía serial (por conexión USB) a una aplicación web. Los datos del ECG (Electrocardiograma) se recogen y procesan para su visualización en tiempo real y se almacenan para análisis posterior mediante el modelo de predicción creadas en la aplicación web que utilizan un conjunto de datos de entrenamiento y de test obtenido de la web Kaggle. Por lo tanto, tenemos que diferenciar entre dos tipos de datos: los recogidos por el sensor que son del paciente y los datos del conjunto de entrenamiento y test utilizados en el modelo de predicción de kaggle \cite{kaggle-data}.

Por un lado, los datos de ECG se obtienen utilizando un sensor (AD8232) conectado a un Arduino. Estos datos se envían a la aplicación web donde se procesan y se muestran en gráficos en tiempo real, también se pueden grabar y descargar en formato Excel para su posterior predicción en el modelo creado que clasifica los latidos cardíacos según los diferentes tipos de latidos que se presentan. Además, una vez hecha la predicción, los resultados se pueden almacenar en una base de datos, la cual se puede descargar para su posterior supervisión por un profesional.

Los datos de ECG que se recogen en el sensor AD8232 se envían a la ventana de escritorio creada para este propósito, que se encuentra en la pagina de datos en vivo de la aplicación web. En la aplicación de escritorio, se puede ver la gráfica con los datos en tiempo real y, si el usuario elige grabar los datos, estos se almacenan en un archivo \textbf{.xls} para después en la ventana de análisis de datos realizar la predicción que clasifica los latidos cardíacos según diferentes tipos de latidos, mediante la segmentación de cada ciclo cardiaco o latido y la comparación con los segmentos del conjunto de entrenamiento que contiene las etiquetas de: Latidos Normales, Latidos de Ectopia Supraventricular, Latidos de Ectopia Ventricular, Latidos de Fusión y Latidos Inclasificables. En la aplicación web se puede visualizar mediante un deslizador cada segmento con la etiqueta que le corresponde y una tabla con todas las predicciones de los segmentos. Además, contiene una herramienta para recortar la grabación en caso de que haya ruido y poder aprovechar una parte de la grabación y no tener que repetir el proceso de nuevo.

También se ha creado una base de datos que tiene como función almacenar las predicciones realizadas por el algoritmo de predicción (RandomForest) siempre que el usuario así lo elija junto a la fecha y hora del momento. También tiene herramientas para borrar en caso de que sea necesario o haya algún error. Esta base de datos define las columnas \textit{id}, \textit{fecha\_hora}, \textit{predicciones\_numeros} y \textit{predicciones\_etiquetas}.


La comunicación entre las diferentes partes del proyecto (Arduino, streamlit y base de datos) se maneja a través de:

\begin{itemize}
\item Un script de Python, encargado de recibir los datos recogidos por el sensor AD8232 conectado a Arduino y enviarlos a la aplicación de escritorio desarrollada con Tkinter. Este script utiliza la librería \texttt{pyserial} para la comunicación serial.
\item Una aplicación web desarrollada con Streamlit, que maneja las peticiones para visualizar y analizar los datos de ECG.
\item Un backend en Python que procesa los datos de ECG y realiza predicciones utilizando un modelo de predicción Random Forest.
\item Una base de datos con SQLite que almacena las predicciones del modelo de predicción Random Forest junto a la fecha y hora.
\end{itemize}

\section{Técnicas y herramientas}

\subsection{Metodologías de desarrollo software}

La planificación y gestión del proyecto se ha realizado utilizando un diagrama de Gantt. Esta herramienta es ideal para visualizar el cronograma y las tareas del proyecto, asegurando el cumplimiento de plazos y la identificación de posibles retrasos.

\subsubsection{Diagrama de Gantt}

Un diagrama de Gantt es una herramienta visual utilizada para planificar y programar proyectos. Este diagrama muestra las tareas del proyecto a lo largo del tiempo y permite a los gestores de proyectos ver la duración de cada tarea, las fechas de inicio y fin, así como las dependencias entre las tareas. El diagrama se compone de dos partes: una lista de tareas en la parte izquierda y una representación gráfica de esas tareas con barras horizontales en la parte derecha \cite{atlassian_gantt}.


En el diagrama de Gantt, las tareas se muestran como barras horizontales a lo largo de una línea de tiempo. La longitud de cada barra representa la duración de la tarea. Las dependencias entre tareas se indican con flechas que conectan las barras correspondientes. Los hitos importantes del proyecto también se pueden marcar en el diagrama para destacar los puntos clave en el cronograma \cite{atlassian_gantt}.

El diagrama de Gantt proporciona una visión clara y concisa del proyecto, mostrando todas las tareas y sus relaciones en un solo lugar, lo que facilita la comprensión del flujo del proyecto y ayuda a identificar posibles conflictos o retrasos. Además, ayuda a planificar y gestionar el tiempo de manera efectiva, permitiendo a los gestores de proyectos asignar recursos de manera óptima y asegurarse de que las tareas se completen dentro de los plazos establecidos. Permite identificar las dependencias entre tareas, lo cual es crucial para planificar de manera efectiva y evitar cuellos de botella. También facilita el seguimiento del progreso del proyecto, permitiendo a los gestores ver qué tareas están en curso, cuáles están completadas y cuáles están retrasadas. Finalmente, el diagrama de Gantt proporciona una herramienta visual que facilita la comunicación del estado del proyecto a todas las partes interesadas, incluyendo equipos de trabajo, clientes y patrocinadores del proyecto.

En este proyecto, el diagrama de Gantt ha sido una herramienta clave para planificar, ejecutar todas las etapas del desarrollo. Desde la investigación inicial hasta la preparación de la presentación final, todas las tareas se han organizado y seguido a través de este diagrama, garantizando que el proyecto se mantuviera en el camino correcto y se completara a tiempo.

El diagrama de Gantt completo, que detalla todas las tareas y su cronograma, se puede encontrar en el anexo B2 de este documento.


\subsection{Entornos de programación y programas}

Para el desarrollo de este proyecto se han utilizado diversas herramientas de software que han facilitado tanto la programación como la gestión de datos y la colaboración. A continuación se detallan las principales herramientas y entornos utilizados:

\begin{itemize}
    \item \textbf{Streamlit:} Framework utilizado para crear la interfaz web de la aplicación. Permite la creación de aplicaciones web interactivas en Python de manera sencilla y eficiente \cite{Streamlit}. Disponible en \url{https://streamlit.io/}.
    \item \textbf{Jupyter Notebook:} Herramienta utilizada para el análisis interactivo de datos y desarrollo de modelos de machine learning. Facilita la documentación y ejecución paso a paso del código, lo que es esencial para la exploración de datos \cite{Jupyter}. Disponible en \url{https://jupyter.org/}.
    \item \textbf{GitHub:} Plataforma utilizada para el control de versiones y la colaboración en el desarrollo del proyecto. Permite el seguimiento de cambios, la gestión de issues y la colaboración con otros desarrolladores \cite{GitHub}. Disponible en \url{https://github.com/}.
    \item \textbf{ChatGPT:} Herramienta de IA muy útil para la resolución de problemas. Accesible desde \url{https://chat.openai.com/}, ha sido útil para el desarrollo de algunas partes del código \cite{ChatGPT}. Disponible en \url{https://openai.com/chatgpt}.
    \item \textbf{Anaconda:} Entorno de distribución de Python utilizado para gestionar paquetes y entornos. Anaconda facilita la instalación y gestión de librerías, además de proporcionar herramientas como Jupyter Notebooks para el desarrollo interactivo \cite{Anaconda}. Disponible en \url{https://www.anaconda.com/}.
    \item \textbf{PowerShell de Anaconda:} Terminal utilizada para ejecutar scripts y notebooks de análisis de datos. Permite una ejecución eficiente y controlada del código, especialmente en el manejo de entornos virtuales y la gestión de dependencias \cite{PowerShell}. Disponible en \url{https://docs.anaconda.com/anaconda/user-guide/tasks/powershell/}.
    \item \textbf{draw.io:} Aplicación utilizada para crear diagramas de casos de uso. Proporciona una interfaz intuitiva y herramientas versátiles para diseñar y organizar diagramas visuales de manera eficiente \cite{drawio}. Disponible en \url{https://draw.io}.
    \item \textbf{Lucidchart:} Aplicación utilizada para crear diagramas de flujo. Facilita la creación de diagramas complejos con una variedad de formas y opciones de personalización, mejorando la visualización y comprensión de los procesos \cite{Lucidchart}. Disponible en \url{https://www.lucidchart.com}.
\end{itemize}

\subsection{Lenguajes de programación}

\begin{itemize}

\item \textbf{Python (v3.9):} Lenguaje principal utilizado para el desarrollo del backend, procesamiento de datos y creación de la interfaz web \cite{Python}.

\item \textbf{JavaScript, HTML y CSS:} Utilizados para el desarrollo de la interfaz de usuario en Streamlit y la aplicación Tkinter \cite{WebTechnologies}.

\item \textbf{SQL:} Utilizado para la gestión de la base de datos SQLite \cite{SQL}.

\item \textbf{Lenguaje Arduino:} El lenguaje de programación utilizado en el entorno Arduino IDE se basa en Wiring y es similar a C++. Este lenguaje permite escribir el código que se cargará en el microcontrolador Arduino UNO R3, facilitando la creación de programas que controlan el hardware de manera eficiente \cite{ArduinoLanguage}.

\end{itemize}

\subsection{Librerías y paquetes}
Para el desarrollo de este proyecto se han utilizado diversas librerías y paquetes de Python, que han facilitado tanto el procesamiento de datos como la implementación del modelo de predicción y la creación de interfaces de usuario. A continuación se detallan las principales librerías utilizadas:


\textbf{Librerías de Arduino:}
Para este programa de Arduino no se ha utilizado ninguna librería externa, solo se han empleado funciones integradas en el entorno de desarrollo de Arduino. Las funciones utilizadas, que pertenecen a la biblioteca estándar de Arduino, son las siguientes:

\begin{itemize}
    \item \texttt{Serial.begin(9600)}
    \item \texttt{Serial.println()}
    \item \texttt{pinMode(pin, mode)}
    \item \texttt{digitalRead(pin)}
    \item \texttt{digitalWrite(pin, value)}
    \item \texttt{analogRead(pin)}
    \item \texttt{millis()}
    \item \texttt{delay(ms)}
\end{itemize}

Por lo tanto, no se necesita incluir ninguna librería adicional explícitamente en el código ya que todas las funciones utilizadas son parte la biblioteca estándar de Arduino.

\subsection{Librerías y paquetes}

Para el desarrollo de este proyecto se han utilizado diversas librerías y paquetes de Python, que han facilitado tanto el procesamiento de datos como la implementación del modelo de predicción y la creación de interfaces de usuario. A continuación se detallan las principales librerías utilizadas:

\textbf{Librerías de Python:}
\begin{itemize}
\item \textbf{Streamlit:} Una biblioteca de Python que permite crear aplicaciones web interactivas de manera sencilla. Es utilizada principalmente para desplegar aplicaciones de ciencia de datos y machine learning \cite{Streamlit}.
\item \textbf{Matplotlib y Seaborn:} Matplotlib es una biblioteca en Python diseñada para crear gráficos en 2D, que abarca gráficos estáticos, animados e interactivos. Seaborn, construida sobre Matplotlib, facilita la generación de gráficos estadísticos visualmente atractivos y con estilos mejorados, optimizando la visualización de datos \cite{matplotlib} \cite{Seaborn}.
\item \textbf{Scipy:} Una biblioteca de Python utilizada para realizar cálculos científicos y técnicos. Incluye módulos para la optimización, integración, interpolación, álgebra lineal, ecuaciones diferenciales y otras tareas científicas \cite{Scipy}.
\item \textbf{Pandas:} Pandas es una biblioteca de Python que ofrece estructuras de datos de alto rendimiento y herramientas de análisis. Es crucial para la manipulación y análisis de datos estructurados, proporcionando funciones para la limpieza, transformación y visualización de datos \cite{Pandas}.
\item \textbf{Joblib:} Una biblioteca para Python que facilita la serialización y deserialización de objetos de Python, especialmente útil para la persistencia de modelos de machine learning y procesamiento paralelo \cite{Joblib}.
\item \textbf{Tkinter:} El paquete de interfaces gráficas estándar de Python. Proporciona herramientas para desarrollar aplicaciones de escritorio con interfaces gráficas de usuario (GUI) \cite{Tkinter}.
\item \textbf{SQLite3:} Un módulo integrado en Python que proporciona una interfaz para trabajar con bases de datos SQLite. Es utilizado para gestionar bases de datos de manera eficiente y sencilla \cite{SQLite}.
\item \textbf{Subprocess y Threading:} Subprocess permite crear nuevos procesos, conectar sus entradas/salidas/error pipes y obtener sus códigos de retorno. Threading facilita la ejecución de tareas en paralelo mediante la creación y manejo de hilos \cite{Python}.
\item \textbf{Scikit-learn:} Scikit-learn es una biblioteca de machine learning en Python que ofrece herramientas eficientes y fáciles de usar para análisis de datos y minería de datos. Incluye algoritmos para clasificación, regresión y clustering, y en este proyecto se utiliza específicamente para realizar predicciones de tipos de ciclos cardíacos del paciente usando el modelo Random Forest \cite{ScikitLearn}.
\end{itemize}

\subsection{Tecnologías de comunicación}
La comunicación entre el microcontrolador Arduino y la aplicación web se realiza a través de una conexión serial, utilizando el puerto USB del dispositivo:

\begin{itemize}
\item \textbf{Comunicación serial:} (Figura \ref{fig:serial_communication})
Utilizada para enviar los datos de ECG desde el Arduino a la aplicación de escritorio y posteriormente a la aplicación web. La comunicación serial es una forma básica pero efectiva de transferir datos entre dispositivos. Funciona enviando datos bit a bit a través de un canal de comunicación, permitiendo que los dispositivos intercambien información de manera eficiente. Este tipo de comunicación puede ser síncrona o asíncrona, siendo la última la más común en sistemas embebidos como Arduino. La comunicación serial asíncrona utiliza un protocolo simple que incluye bits de inicio, datos y parada para asegurar que los datos sean interpretados correctamente por el dispositivo receptor \cite{serial_communication_basics}.
\end{itemize}

\begin{figure}[h]
\centering
\includegraphics[width=0.5\textwidth]{img/Serial.png}
\caption{Esquema de comunicación serial \cite{serial_communication_basics}.}
\label{fig:serial_communication}
\end{figure}



\subsection{Herramientas Hardware}

\begin{itemize}

\item \textbf{Microcontrolador arduino UNO R3:} (Figura \ref{fig:Arduino})
La placa Arduino UNO R3 está basada en el microcontrolador ATMega328P. Tiene pines para entradas y salidas tanto digitales como analógicas, lo que permite conectarla a varios sensores y dispositivos. Es fácil de programar usando el Arduino IDE y permite cargar programas a través de una conexión USB \cite{ArduinoUNO3}.

\begin{figure}[h]
\centering
\includegraphics[width=0.5\textwidth]{img/arduinouno.jpg}
\caption{Microcontrolador Arduino UNO R3 \cite{ArduinoUNO3}.}
\label{fig:Arduino}
\end{figure}
\end{itemize}

\begin{itemize}
\item \textbf{Cable USB:} (Figura \ref{fig:usb
})
Conector para cargar el programa en el microcontrolador de Arduino a través del ordenador. Es fundamental para la carga de programas y la comunicación entre la placa y el entorno de desarrollo Arduino IDE. El cable USB también proporciona alimentación a la placa durante la programación y el desarrollo de proyectos. En este caso también tiene la función de enviar los datos recogidos por el sensor a través del puerto COM al que esta conectado y poder visualizarlo en la aplicación de escritorio en tiempo real.


\begin{figure}[h]
\centering
\includegraphics[width=0.5\textwidth]{img/cableUSB.jpg}
\caption{Cable para Conexión USB-Arduino \cite{CableUSB}.}
\label{fig:usb
}
\end{figure}
\end{itemize}


\begin{itemize}
\item \textbf{Cables:} (Figura \ref{fig:Cables
})
Componentes necesarios para conectar el sensor de ECG al Arduino.

\begin{figure}[h]
\centering
\includegraphics[width=0.5\textwidth]{img/cablesarduino.jpg}
\caption{Cables \cite{Cables}.}
\label{fig:Cables
}
\end{figure}
\end{itemize}

\begin{itemize}
\item \textbf{Electrodos:} (Figura \ref{fig:electrodos})
Componentes necesarios para conectar el sensor de ECG al cuerpo del paciente, asegurando una buena calidad de señal. Estos electrodos son adhesivos y desechables, diseñados para adherirse a la piel del paciente y proporcionar una conexión estable y confiable. La calidad de los electrodos es crucial para obtener lecturas precisas y minimizar el ruido en los datos de ECG. Generalmente están fabricados con materiales conductores que permiten captar las señales eléctricas generadas por el corazón y transmitirlas al sensor de ECG \cite{ElectrodosECG}.

\begin{figure}[h]
\centering
\includegraphics[width=0.5\textwidth]{img/electrodo.jpg}
\caption{Electrodos \cite{ElectrodosECGPediatricos}.}
\label{fig:electrodos}
\end{figure}
\end{itemize}


\subsection{Comparación de diferentes sensores a emplear}

\begin{itemize}
\item \textbf{Módulo AD8232:} (Figura \ref{fig:ad8232})
El módulo AD8232 es un dispositivo que se utiliza para medir señales de ECG y otros biopotenciales. Su función es capturar, aumentar y limpiar las señales eléctricas del cuerpo, incluso en entornos ruidosos. Este módulo facilita que un convertidor analógico-digital (ADC) de bajo consumo o un microcontrolador puedan leer fácilmente la señal de salida \cite{AD8232_teoria}.

\begin{figure}[h]
\centering
\includegraphics[width=0.5\textwidth]{img/ad8232.jpg}
\caption{Módulo AD8232 \cite{AD8232}.}
\label{fig:ad8232}
\end{figure}

\item \textbf{Sensor KY-039:} (Figura \ref{fig:ky039})
El sensor KY-039 es un módulo de pulso cardíaco que se basa en un emisor y receptor de luz infrarroja para detectar el pulso cardíaco. Funciona al medir la variación en la luz reflejada o absorbida por el flujo sanguíneo en el dedo del usuario. Aunque no es tan preciso como el módulo AD8232 para medir ECG, es una opción más económica y fácil de usar para aplicaciones menos críticas donde se requiere la detección básica del ritmo cardíaco \cite{KY039_teoria}.

\begin{figure}[h]
\centering
\includegraphics[width=0.5\textwidth]{img/ky-039.jpg}
\caption{Sensor KY-039 \cite{KY039}.}
\label{fig:ky039}
\end{figure}

\item \textbf{Módulo MAX30100:} (Figura \ref{fig:max30100})
El módulo MAX30100 es un sensor de pulso y oxímetro de alta precisión. Combina un pulsioxímetro y un sensor de ritmo cardíaco. Es ideal para aplicaciones que requieren monitoreo de salud a nivel profesional. Necesita una tensión de entrada de 1.8V y es capaz de medir la saturación de oxígeno en sangre junto con la frecuencia cardíaca \cite{MAX30100_teoria}.

\begin{figure}[h]
\centering
\includegraphics[width=0.5\textwidth]{img/max30100.jpg}
\caption{Módulo MAX30100 \cite{MAX30100}.}
\label{fig:max30100}
\end{figure}

\end{itemize}

\subsection{Conclusión}

En conclusión, he elegido el módulo AD8232 por su alta precisión y capacidad para manejar señales ECG, lo cual es muy importante para tener mediciones fiables y detalladas del ECG. Aunque el sensor KY-039 es más económico y fácil de usar, su precisión es bastante baja y está más orientado a aplicaciones donde solo se necesita la detección del ritmo cardíaco. El módulo MAX30100, aunque ofrece la ventaja de medir la saturación de oxígeno en sangre, es más costoso, por lo que el AD8232 sigue siendo la mejor opción para el balance entre precisión, costo y facilidad de uso en este proyecto.

El AD8232 es capaz de extraer, amplificar y filtrar señales biopotenciales pequeñas, lo que lo hace ideal para nuestro proyecto de monitoreo cardíaco en tiempo real, donde la exactitud y la fiabilidad de los datos son la clave de todo el proyecto. Además, tiene la capacidad de integrarse fácilmente con un microcontrolador. En resumen, aunque el costo del AD8232 es más alto, su precisión y capacidad para operar en condiciones difíciles justifican su elección para el proyecto (ver Tabla \ref{tab:camparacion}).

\begin{table}[h]
	\centering
	\rowcolors{2}{gray!25}{white}
	\begin{tabularx}{\linewidth}{ p{0.3\columnwidth} p{0.4\columnwidth} p{0.2\columnwidth} }
		\toprule
		\textbf{Módulos} & \textbf{Características} & \textbf{Precio} \\
		\toprule
		\textbf{Módulo AD8232} & Mide señales ECG, alta precisión, adecuado para condiciones ruidosas & 10-20€ \\
		\textbf{Sensor KY-039} & Detección básica del ritmo cardíaco mediante luz infrarroja & 3-10€ \\
		\textbf{Módulo MAX30100} & Mide frecuencia cardíaca y saturación de oxígeno en sangre, alta precisión & 15-25€ \\
		\bottomrule
	\end{tabularx}
	\caption{Resumen y comparación de posibles sensores a emplear en el proyecto.}
	\label{tab:camparacion}
\end{table}


\subsection{Mejoras hardware}

He implementado unas pequeñas mejoras en el prototipo para añadir comodidad al paciente y un mejor entendimiento. Se ha añadido un LED que se enciende por cada latido del corazón, lo cual nos puede ayudar a ver que todo está correctamente conectado y en funcionamiento antes de su puesta en marcha en la aplicación web. Además, para una mayor comodidad y orden con los diferentes cables del prototipo, he diseñado una caja para imprimirla en una impresora 3D y que el Arduino y todo el cableado se encuentre en su interior. Hardware empleado para las mejoras:

\begin{itemize}
\item \textbf{Caja en 3D:} (Figura \ref{fig:Caja
})
Esta fabricada con una impresora 3D con material PLA+, tiene como principal función proteger el Arduino, las conexiones entre todos los componentes y sobretodo mantener ordenados los cables que están conectados al Arduino. La caja cuenta con perforaciones diseñadas para:
\begin{itemize}
\item Visualizar del LED.
\item Conexión USB del microprocesador Arduino UNO para cargar los programas y conexión con la aplicación web.
\item Organización de cables para mantener un aspecto ordenado.
\end{itemize}

\begin{figure}[h]
\centering
\includegraphics[width=0.5\textwidth]{img/caja3d.jpg}
\caption{Caja para Prototipos fabricada mediante impresión 3D (Fuente propia).}
\label{fig:Caja
}
\end{figure}
\end{itemize}

\begin{itemize}
\item \textbf{LED para Marcar el Pulso del Paciente:} (Figura \ref{fig:LED
})
Conectado al pin 13 (salida digital de Arduino), el LED parpadea en sincronía con el pulso del paciente, proporcionando una indicación visual del ritmo cardíaco y también tiene la función de indicar que todo está funcionando correctamente de cara a realizar la grabación de los datos para su posterior análisis.

\begin{figure}[h]
\centering
\includegraphics[width=0.5\textwidth]{img/ledverde.jpg}
\caption{LED para Marcar el Pulso del Paciente \cite{LED}.}
\label{fig:LED
}
\end{figure}
\end{itemize}

\begin{itemize}
\item \textbf{Resistencias para el LED:} (Figura \ref{fig:Resistencias
})
Utilizadas para limitar la corriente que pasa a través del LED y otros componentes, protegiendo el circuito de posibles daños. Estas resistencias se calcularon para asegurar un funcionamiento óptimo y seguro.

\begin{figure}[h]
\centering
\includegraphics[width=0.5\textwidth]{img/resistencia.jpg}
\caption{Resistencias necesarias para el correcto funcionamiento del circuito \cite{Resistencia}.}
\label{fig:Resistencias
}
\end{figure}
\end{itemize}
































\capitulo{5}{Resultados}

\section{Resumen de resultados}

El desarrollo del sistema de monitoreo cardíaco en tiempo real junto al modelo de predicción ha permitido la creación de una herramienta efectiva para la supervisión continua de la actividad cardíaca. Los objetivos establecidos al inicio del proyecto se han cumplido de la siguiente manera:

1. \textbf{Conexión estable vía USB}: La comunicación entre el Arduino y la aplicación web se realiza mediante una conexión USB, la cual quizás no es la más óptima en cuanto a comodidad y manejabilidad, pero viendo el lado positivo se elimina la necesidad de baterías y se garantiza una transmisión de datos constante y fiable.

2. \textbf{Monitoreo continuo de la actividad cardíaca}: El sistema desarrollado permite la visualización en tiempo real de los datos de ECG, proporcionando una herramienta útil para la detección temprana de irregularidades cardíacas.

3. \textbf{Accesibilidad y facilidad de uso}: La interfaz web creada con Streamlit es intuitiva y accesible, permitiendo a los usuarios sin conocimientos técnicos manejar el sistema con facilidad (Figura \ref{fig:interfaz}).

4. \textbf{Predicción del tipo de ciclo cardíaco}: Utilizando modelos de machine learning, el sistema puede realizar predicciones que clasifican cada ciclo cardíaco (latido de paciente), pudiendo así detectar alguna anomalía, aunque la precisión está limitada por la calidad de los datos de entrenamiento disponibles (Figura \ref{fig:analisis}).

Aun asi la parte positiva es que conocer los tipos de latidos del paciente nos da información valiosa para la evaluación y gestión de la salud cardíaca. A continuación, se detallan algunos beneficios específicos:

Identificar patrones anormales en los latidos, como latidos de ectopia supraventricular o ventricular, puede ayudar en el diagnóstico temprano de arritmias y otras condiciones cardíacas.
Detectar y clasificar latidos anormales puede prevenir complicaciones graves como infartos de miocardio, fibrilación auricular y otras arritmias potencialmente peligrosas.

5. \textbf{Almacenamiento y análisis de datos}: Se ha implementado una base de datos SQLite para almacenar las predicciones, facilitando que se pueda descargar en un archivo \textbf{.csv} para la generación de estadísticas (Figura \ref{fig:bbdd}).


\begin{figure}[h]
\centering
\includegraphics[width=0.8\textwidth]{img/proyecto.jpg}
\caption{Hardware del sistema de monitoreo cardíaco, mostrando el Arduino conectado al sensor de ECG (Fuente propia).}
\label{fig:hardware}
\end{figure}

\begin{figure}[h]
\centering
\includegraphics[width=0.8\textwidth]{img/interfaz_streamlit.png}
\caption{Interfaz de la aplicación web creada en Streamlit (Fuente propia).}
\label{fig:interfaz}
\end{figure}

\begin{figure}[h]
\centering
\includegraphics[width=0.8\textwidth]{img/ventana_tinker.png}
\caption{Ventana de visualización y grabación de datos en vivo en la aplicación web (Fuente propia).}
\label{fig:ventana}
\end{figure}

\begin{figure}[h]
\centering
\begin{minipage}[b]{0.45\textwidth}
\centering
\includegraphics[width=\textwidth]{img/analisis1.png}
\caption{Apartado donde se detectan los picos R de cada ciclo cardíaco para segmentar y realizar la predicción de datos de ECG introducidos en la aplicación web (Fuente propia).}
\label{fig:analisis1}
\end{minipage}
\hfill
\begin{minipage}[b]{0.45\textwidth}
\centering
\includegraphics[width=\textwidth]{img/analisis2.png}
\caption{Tabla con los resultados de la predicción que clasifica cada tipo de ciclo cardíaco de la muestra introducida en la aplicación web (Fuente propia).}
\label{fig:analisis2}
\end{minipage}
\caption{Ventana de análisis de datos de ECG en la aplicación web.}
\label{fig:analisis}
\end{figure}

\begin{figure}[h]
\centering
\includegraphics[width=0.8\textwidth]{img/bbdd.png}
\caption{Ventana de almacenamiento de las predicciones en la base de datos de la aplicación web (Fuente propia).}
\label{fig:bbdd}
\end{figure}

El prototipo puede llegar a ser una herramienta prometedora para el monitoreo cardíaco, pero hay áreas que requieren mejoras adicionales, especialmente en la precisión de las predicciones y la calidad de los datos que utiliza el modelo como entrenamiento.

\section{Discusión}

Los resultados de este proyecto representan un avance significativo en el campo del monitoreo cardíaco en tiempo real. A continuación, se discuten los aspectos más destacados y las áreas de mejora, basados en la implementación y los resultados obtenidos.

Uno de los logros más importantes es la creación de una interfaz web intuitiva y accesible mediante Streamlit, que permite a los usuarios sin conocimientos técnicos supervisar la actividad cardíaca en tiempo real (Figura \ref{fig:ventana}). La facilidad de uso y la capacidad de visualización en tiempo real proporcionan una herramienta valiosa tanto para profesionales de la salud como para pacientes.

La implementación de un sistema de predicción basado en machine learning es otro avance clave. Utilizando modelos de Random Forest, el sistema puede clasificar cada ciclo cardíaco y detectar anomalías, lo cual es esencial para la prevención y diagnóstico temprano de enfermedades cardíacas. Como se detalla en la sección de resultados, conocer y clasificar los tipos de latidos cardíacos es crucial para un manejo integral de la salud cardíaca (Figura \ref{fig:analisis}). Esto permite diagnósticos más precisos, tratamientos personalizados y una mejor prevención de enfermedades cardíacas graves.

Sin embargo, se identificaron varias áreas que requieren mejoras adicionales. La precisión del modelo de predicción está limitada por la calidad y cantidad de los datos de entrenamiento disponibles. Es fundamental mejorar la calidad de los datos y aumentar el conjunto de datos de entrenamiento para mejorar la robustez y precisión del modelo.

Además, aunque la conexión USB utilizada garantiza una transmisión de datos fiable, no es la opción más cómoda para el usuario. Implementar una conexión inalámbrica estable, como Bluetooth o WiFi, sería un paso importante para mejorar la comodidad y movilidad del sistema.

Para evaluar la facilidad de uso de la aplicación web creada, se utilizó la Escala de Usabilidad del Sistema (SUS). La SUS es una herramienta sencilla que, mediante un cuestionario de diez preguntas, permite evaluar rápidamente distintos aspectos de la usabilidad de la web. Este cuestionario examina puntos como la facilidad de uso, la eficiencia y la satisfacción del usuario, lo que hace que los resultados y las conclusiones obtenidas sean más válidos y confiables \cite{Brooke}. Realizar esta evaluación proporciona una visión de la efectividad y usabilidad del proyecto en general y ayuda mucho a identificar posibles áreas de mejora. La encuesta y sus resultados pueden consultarse en el Anexo G-Estudio experimental

En conclusión, en este proyecto se ha desarrollado un sistema de monitoreo cardíaco en tiempo real que ofrece herramientas para el análisis de los datos. Aunque se han cumplido los objetivos principales, la implementación de mejoras en la calidad de los datos, la conexión inalámbrica y la evaluación de la usabilidad contribuirá significativamente a la eficacia y utilidad del sistema

\capitulo{6}{Conclusiones}

El desarrollo del sistema de monitoreo cardíaco en tiempo real ha logrado cumplir con los objetivos establecidos al inicio del proyecto. Este proyecto ha permitido la creación de una herramienta innovadora, efectiva y económica para la supervisión continua de la actividad cardíaca.

Desde la implementación inicial hasta los resultados obtenidos, el sistema ha demostrado ser una solución viable y práctica para el monitoreo cardíaco, proporcionando datos en tiempo real que pueden ser utilizados tanto por pacientes como por profesionales de la salud. La elección del módulo AD8232, debido a su alta precisión y capacidad para manejar señales ECG en condiciones ruidosas, ha sido un factor clave en la eficacia del sistema.

\section{Aspectos relevantes}

Uno de los mayores desafíos enfrentados durante el desarrollo del proyecto fue establecer una conexión inalámbrica eficiente entre el dispositivo Arduino y la aplicación web. Inicialmente, se consideró la implementación de un módulo Bluetooth o WiFi, sin embargo, se optó por una conexión USB debido a problemas de transmisión de los datos en las pruebas realizadas con la conexión inalámbrica. Este cambio forzado, aunque limitó la movilidad, garantizó una transmisión de datos constante y fiable.

El desarrollo de la interfaz web con Streamlit presentó una serie de desafíos debido a la falta de experiencia realizando este tipo de trabajos pero observando algunos ejemplos y documentandome sobre su uso se consiguió un resultado final bueno y sencillo. La implementación de una base de datos SQLite y el uso de modelos de machine learning para la predicción de anomalías cardíacas representaron importantes avances que enriquecieron el proyecto.

\section{Limitaciones y desafíos}

El proyecto encontró varias limitaciones, especialmente en términos de la calidad de los datos de entrenamiento para el modelo de predicción. La precisión del modelo de machine learning está directamente influenciada por la calidad y diversidad de los datos utilizados, lo que resalta la necesidad de mejorar los conjuntos de datos de entrenamiento en futuras implementaciones.

Además, aunque la conexión USB proporcionó una solución estable, la falta de movilidad sigue siendo una limitación significativa. Retomar la investigación para implementar una conexión inalámbrica efectiva sigue siendo una prioridad para mejorar la comodidad y la usabilidad del sistema.



\capitulo{7}{Líneas de trabajo futuras}

Aunque el proyecto ha logrado sus objetivos, existen oportunidades para futuras mejoras e implementaciones en diversas áreas.

En primer lugar, es crucial mejorar la precisión del modelo de predicción. Para ello, es necesario aumentar la calidad y cantidad de los datos de entrenamiento, utilizando bases de datos más extensas y diversificadas, lo que permitirá mejorar la precisión del modelo de machine learning. Además, se debe permitir que el modelo pueda clasificar y añadir etiquetas de patologías directamente, no solo clasificar los tipos de latido de cada segmento.

Otra área de mejora es la implementación de una conexión inalámbrica estable. Es necesario retomar la investigación y desarrollo para implementar un módulo Bluetooth o WiFi que permita la comunicación inalámbrica entre el dispositivo de monitoreo y la aplicación web. Esto mejorará la comodidad y movilidad del usuario, eliminando la dependencia de la conexión USB.

La ampliación de las funcionalidades de la aplicación web también es una prioridad. Integrar herramientas adicionales para el análisis de los datos de ECG, como la detección automática de diferentes tipos de arritmias, y generar informes detallados para los profesionales de la salud facilitará el diagnóstico y seguimiento del paciente.

En cuanto al hardware, se deben desarrollar versiones más compactas y ergonómicas del dispositivo de monitoreo. Además, incluir opciones para la alimentación mediante baterías recargables aumentará la autonomía del sistema.

Finalmente, es esencial implementar medidas adicionales para asegurar la privacidad y seguridad de los datos de los pacientes. Cumplir con normativas y estándares internacionales de protección de datos garantizará la confidencialidad de la información médica.





\bibliographystyle{unsrt}
\bibliography{bibliografia}

\end{document}
